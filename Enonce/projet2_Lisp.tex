\documentclass{../../../LaTeX/tdsimple}

\usepackage[utf8]{inputenc}%
\usepackage[francais]{babel}%
\usepackage{listings}%
\usepackage{url}%

\usepackage{wasysym}%
\usepackage{ifthen}%

\begin{document}

\title{ENS Rennes\\
  L3 Informatique, parcours R \& I\\
  Programmation avancée, 2\ieme\ semestre 2015--2016\\
  ---\\
  Projet 2: Interprète Lisp en C++%
}

\author{%
  Luc Bougé%
}

\date{4 avril 2016%
  \thanks{%
    \protect\lstinline[language={}]%
    $Id: projet2_Lisp.tex 2052 2016-04-03 19:36:04Z bouge $%
  } }

% svn propset svn:keywords Id *.tex

\maketitle

\section*{Introduction}

\subsection*{Présentation du projet}

Le but de ce projet est de faire une synthèse de l'ensemble de votre
formation en programmation cette année. Vous avez étudié dans vos
divers cours une grande variété de langages: Python, Caml, Scala, C,
C++, Lisp, Scheme, etc. Vous avez aussi été exposés à une grande
variété d'approches de la programmation, ce qui est très bien.

Dans la plupart des cours, chaque langage et son approche associée
ont été étudiées spécifiquement. L'objectif est ici de vous proposer
un projet qui mixe plusieurs approches: C++ et Lisp/Scheme, en prenant
appui sur Caml. Vous aurez aussi l'occasion d'utiliser les outils Lex/Flex et
Yacc/Bison qui sont une application directe de votre cours de théorie des
langages. 

Vous connaissez déjà les références pour C++ et Lisp/Scheme. Il existe
de très nombreux tutoriels pour Lex/Flex et Yacc/Bison. Nous allons
ici utiliser la version C++ de ces outils. Un tutoriel rapide pour
l'utilisation \emph{conjointe} de Flex et Bison est disponible à
l'adresse
\begin{center}
  \url{http://aquamentus.com/flex_bison.html}
\end{center}
Flex et Bison sont des outils GNU avec une documentation abondante à
l'adresse habituelle.

  \subsection*{Déroulement et évaluation}

  Le sujet est composé d'une partie obligatoire commune et d'un choix
  de parties optionnelles ouvertes et graduées. Il est demandé de
  choisir l'une des parties optionnelles, pas plus. Par contre, un
  grand soin est attendu dans le travail de conception et de
  validation.

  Vous avez le droit de vous inspirer et même le devoir de reprendre
  les codes que vous avez écrits dans les séances de TP
  précédentes. Vous êtes encouragés à utiliser vos notes de cours,
  tout document et tout livre à votre convenance, et à consulter sur
  Internet la documentation C++ et Lisp/Scheme et tout site qui vous
  paraîtrait utile.

  Ce projet a été conçu entièrement cette année. Je vous remercie par
  avance de vos commentaires et suggestions!  Malheureusement, il est
  difficile d'évaluer le temps nécessaire. Ce sera ajusté avec vous au
  cours du projet.

  \section{Présentation du code préparé}

  \begin{attention}
    L'objectif est ici la \emph{robustesse} du codage, pas la
    performance. Toute la phase de codage doit être traitée de manière
    \emph{défensive} en utilisant des assertions à chaque fois
    que c'est possible. Utilisez largement les variables
    intermédiaires pour être sûrs de vos manipulations. Ne cherchez surtout
    pas à optimiser, mais assurez-vous de la correction de vos
    fonctions pas à pas.
  \end{attention}

  Le code C++ qui est préparé implémente une forme très rudimentaire
  d'un interprète à liaison dynamique, sur le modèle de l'interprète
  \file|MLlisp_dynamic| que nous avons utilisé en cours.

  Les principes de structuration des codes C++ ne sont pas les mêmes
  que ceux de Caml. Ce sera à vous de bien distribuer votre code entre
  différents fichiers et modules.

\subsection{Cellules et objets Lisp}

L'utilisation de C++ demande une certaine virtuosité pour la mise au
point de la classe \|Cell| qui est l'unité de base de l'allocation
Lisp. En effet, une cellule peut contenir un nombre, une chaîne, un
symbole ou une paire de pointeurs vers des cellules. C'est donc une
\|union| C++. Malheureusement, dans la version du langage que nous
utilisons, les \|union| ne peuvent pas contenir des types avec
constructeurs, en particulier \|string|. Il faut donc stocker les
chaînes et les symboles sous la forme de tableau de caractères, comme
en C.
\begin{quotation}
  \url{https://en.wikipedia.org/wiki/Union_type}

  C++ (since C++11) also allows for a data member to be any type that
  has a full-fledged constructor/destructor and/or copy constructor,
  or a non-trivial copy assignment operator. For example, it is
  possible to have the standard C++ string as a member of a union.
\end{quotation}

Nous devons donc remplacer les \|string| C++ par des tableaux de
caractères (des pointeurs) alloués explicitement dans le tas par
\|malloc| dans la fonction \|strdup|. Ce point sera important dans
l'étude fine de l'algorithme de gestion de la mémoire (\emph{garbage
  collection}). Vous trouverez dans le fichier \file|cell.cc| la
définition de la classe \|Cell| et des méthodes qui permettent de
créer une cellule à partir d'un objet d'un certain type (\|int|,
\|string|, etc.)  et dans le sens contraire de retrouver l'object
original d'une cellule. Tous ces traitements délicats sont entourés
d'assertions sur un mode très défensif.

La classe des objets Lisp est la classe \|Object| définie dans le
fichier \|object.cc|. L'adresse d'un objet \|static| de la classe \|Cell| est
utilisé comme la valeur de \|nil|. Les fonctions usuelles sont
définies sur les objets. 

\subsection{Évaluateur}

Un forme simple de l'évaluateur Lisp est programmée dans le fichier
\file|eval.cc|.  La récursivité mutuelle des fonctions \|eval| et \|apply|
est plus simple à traiter qu'en Caml. Une trace rudimentaire est mise
en place pour ces deux fonctions.

Les erreurs sont traitées par des appels à des exceptions C++ dérivées
de la classe \|runtime_error|.

Le traitement des \emph{subr} est minimal: seuls \|+| et \|*| sont
définis et leurs définitions sont incluses directement dans le code de
l'évaluateur.

\subsection{Environnement}

Les environnements sont gérés par la classe \|Environnement| définie
dans le fichier \file|env.cc|. Un environnement est ici implémenté par
une liste au sens C++ d'objets Lisp, c'est-à-dire de pointeurs vers
des cellules. À chaque fois que l'on passe sous un $\lambda$, une
copie de l'environnement est faite, ce qui est bien sûr peu
efficace. Ce sera un point à améliorer, ô futurs chercheurs!

Les erreurs sont traitées par des appels à des exceptions C++ dérivées
de la classe \|runtime_error|.

\subsection{Lecture}

Les lectures sont faites par un \emph{lexer} et un \emph{parser}
construits par les outils Flex et Bison. Toute erreur de syntaxe est
fatale. Les chaînes (entre guillemets), les symboles et les entiers
sont reconnus et le \emph{quote} est expansé: \|'(1 2 3)| est lu
comme %
\|(quote (1 2 3)|. Le symbole \|nil| est transformé à la volée en la
liste vide.

\begin{attention}
  Le \emph{parser} \file|lisp.y| n'est pas complètement au point par
  manque de temps de ma part. Ce sera donc à vous de le
  compléter!~\smiley
\end{attention}

\subsection{Toplevel}

Il n'y a pas de \emph{toplevel}. À vous de le faire en vous inspirant
de l'interprète Caml.
\begin{attention}
  Vous aurez en particulier à gérer le rattrapage des exceptions
  lancées par l'évaluateur et l'environnement.
\end{attention}

\section{Partie obligatoire: interprète à liaison dynamique}

Dans un premier temps, il vous est demandé de compléter ce code pour
obtenir un interprète du niveau de l'interprète \file|MLlisp_dynamic|
que nous avons utilisé en cours. Voici quelques étapes de base, mais
vous pouvez tout à fait en proposer d'autres.
\begin{description}

\item[\emph{Toplevel}.]  Construisez un \emph{toplevel} qui reconnaît
  la directive \|setq| et qui lance l'évaluation. Les erreurs
  d'évaluation doivent être rattrapées, signalées à l'utilisateur et
  traitées.

\item[\emph{subr}.] Implémentez une gestion des fonctions \emph{subr}
  intégrée dans l'environnement comme dans l'interprète du
  cours. Ajoutez les fonctions manquantes, en particulier \|read|, etc.
  \begin{attention}
    Ceci va nécessiter de revoir les classes de base pour pouvoir
    manipuler les \emph{subr} dans le monde Lisp.
  \end{attention}

\item[Trace.] Ajoutez un dispositif de trace et d'impression
  (\emph{pretty-printing}) pour aider à la mise au point.

\item[Démonstration.] Montrer que toutes les fonctions habituelles
  fonctionnent bien. Comparez la vitesse de votre interprète avec
  celui de l'interprète du cours.

\end{description}

\section{Partie optionnelle: extensions}

\newcounter{starcounter}

\newcommand{\starlevel}[1]{%
\setcounter{starcounter}{0}%
\whiledo {\value{starcounter} < #1}{%
$\star$%
\stepcounter{starcounter}%
}%
}

\subsection{Méta-interprétation \protect\starlevel{1}}

Enrichissez votre interprète pour pouvoir exécuter l'interprète
méta-circulaire Scheme 
\begin{center}
  \file|ch4-mceval.scm|
\end{center}
du livre SICP à l'adresse
\url{https://mitpress.mit.edu/sicp/code/}. 
\begin{attention}
  Cet interprète est écrit en Scheme, il faudra donc le modifier
  légèrement pour qu'il fonctionne dans un interprète à liaison
  dynamique.
\end{attention}

\subsection{Amélioration de l'interprète \protect\starlevel{2}}

Implémentez une gestion d'environnement par liste chaînée. Adaptez
votre interprète pour passer en liaison statique. Implémentez la
gestion par \emph{frames} de Scheme.  Implémentez les clôtures Scheme.

\subsection{Continuations et futurs \protect\starlevel{3}}

Implémentez la construction \|call/cc| et la construction \|delay| de
Scheme. Vous pouvez le faire sur l'interprète Caml ou C++ à votre
gré. Ceci nécessite l'utilisation des clôtures. Utilisez le livre SICP
pour bien comprendre la sémantique de ces constructions.

\subsection{Gestion mémoire \protect\starlevel{4}}

Dans la version actuelle, les cellules sont créés par la construction
\|new|. Créez une classe \|Memory| qui fonctionne comme un réservoir
de cellules.
\begin{attention}
  La class \|Memory| doit gérer un vecteur statique de cellules.  La
  méthode statique \|allocate| renvoie l'adresse d'une cellule non
  encore utilisée, comme pour la fonction \|malloc|. Votre code ne
  doit plus comporter de \|new Cell|. Voici par exemple la nouvelle
  version de la fonction \|cons|:
\begin{lstlisting}[language=c++]
Object cons(Object a, Object l) {
  // Object p = new Cell();
  Object p = Memory::allocate();
  p -> make_cell_pair(a, l);
  return p;
}
\end{lstlisting}
\end{attention}

Quand la mémoire est saturée, il faut libérer de la place en faisant
le ménage (\emph{house keeping}). L'idée est de repérer toutes les
cellules qui ne sont plus atteignables depuis le \emph{topelevel} et de
les \emph{recycler} en des cellules à nouveau utilisables. Vous pouvez lire
la section~5.3.2 de SICP pour une introduction. 
\begin{attention}
  La mise au point d'un \emph{garbage collector} (GC) est 
  délicate. Il faut adopter une discipline de programmation très
  défensive pour être sûr de ne pas faire d'erreur sur le statut des cellules.
\end{attention}

La référence classique dans ce domaine est le livre \emph{The Garbage
  Collection Handbook: The Art of Automatic Memory Management}, par
Richard Jones, Antony Hosking, Eliot Moss, CRC Press, 2011. La
première version date de 1996. Il existe de très nombreux cours sur le
Web sur ce sujet. Voir par exemple la page Wikipédia \emph{Garbage
  Collection}.
\begin{center}
  \url{https://en.wikipedia.org/wiki/Garbage_collection_%28computer_science%29}
\end{center}

\end{document}

%%% Local Variables:
%%% mode: latex
%%% TeX-master: t
%%% End:
